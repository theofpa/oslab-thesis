Μια καλή δουλειά, πρέπει να προγραμματιστεί. Αν ξεκινάει κανείς απλά χωρίς να παίρνει υπόψη για παράδειγμα την ομάδα- στόχος, το βαθμό λεπτομέρειας και το περίγραμμα, συνήθως δεν θα οδηγηθεί στο αποτέλεσμα. Αυτό ισχύει ιδιαίτερα για τα μεγάλα έργα, όπως διατριβές. Αλλά και για μια δεκασέλιδη εργασία θα πρέπει ακριβώς ένας άπειρος συγγραφέας προηγουμένως να καταστρώσει ένα ολοκληρωμένο σχέδιο της.
Το κύριο μέρος μιας ιδέας είναι το “ακατέργαστο” περίγραμμα. Καταγράψτε τους τίτλους των τμημάτων του έργου σας. Γράψτε στη συνέχεια, για κάθε στοιχείο που δείχνει το περίγραμμα το βασικό περιεχόμενο κάθε τμήματος κι αυτό σε μια απλοποιημένη μορφή πχ. αριθμημένης λίστας ή κουκκίδων. Μ' αυτόν το τρόπο αναπτύξτε το κεντρικό θέμα του έργου, τη ροή της σκέψης που θέλετε να μεταφέρετε στον αναγνώστη. Ενδεχομένως θα συνειδητοποιήσετε ότι ίσως η αρχική δομή θέλει προσαρμογή που οφείλεται στο ότι οι διευκρινίσεις που προστίθενται στα κυριότερα σημεία εμφανίζουν νέες πτυχές της εργασίας. Αυτό δεν είναι κακό. Μάλλον το αντίθετο γιατί προλαμβάνετε έτσι μια επίπονη εργασία συνολικής αναθεώρησης του έργου σας στα επόμενα στάδια της δημιουργίας του.
Βάλτε τον εαυτό σας στη θέση των αναγνωστών τους (μη ξεχνάτε την ομάδα-στόχο). Μπορεί κανείς να ακολουθήσει την επιχειρηματολογία σας εύκολα; Υπάρχουν ξαφνικά άλματα της σκέψης που κάνουν δύσκολη την ανάγνωση; Προσαρμόστε το ``χονδροειδές'' περίγραμμα ξανά και ξανά, μέχρι να έχει νόημα για εσάς. Ταυτόχρονα μπορείτε να ολοκληρώσετε τα βασικά σημεία με στοιχεία που λείπουν στο περιεχόμενο.
Το πλεονέκτημα αυτής της προσέγγισης είναι ότι μπορείτε με μια ματιά να δείτε ολόκληρο το έγγραφο, χρησιμοποιώντας το “ακατέργαστο” περίγραμμα. Στο μυαλό σας διαμορφώνονται ήδη τμήματα κειμένου. Τώρα μπορείτε πολύ γρήγορα κάνετε εκτεταμένες αλλαγές στο σκελετό. Αργότερα, κατά τη συγγραφή του κειμένου, μια τέτοια μετάβαση είναι πιο δύσκολη. Κατά συνέπεια, η δομή πρέπει να είναι σωστή πριν ξεκινήσετε με το κείμενο. Η χονδροειδές  περίγραμμα για μια εργασία εργαστηρίου ή τελική έκθεση έργου θα πρέπει να έχει έκταση περίπου μιας σελίδας. Για τις πτυχιακές εργασίες υφίστανται πολλές σελίδες.
Κατά τη συγγραφή του τελικού κειμένου, μπορείτε πλέον μετατρέπετε τα βασικά σημεία της αρχικής ανάλυσης σε τρέχον κείμενο. Μπορείτε να καθοδηγείστε ανά πάσα στιγμή απ' αυτό το σκελετό. Εάν είστε στη μέση του κειμένου, μπορείτε πάντα να ξέρετε τι υπήρχε πριν και τι θα έρθει μετά απ' αυτό.
Η προσέγγιση αυτή μπορεί να σας  εμφανίζεται για περιττή, αλλά σας βοηθά να οργανώσετε τις σκέψεις σας. Θα διαπιστώσετε ότι γράφετε ευκολότερα και τα κείμενά σας θα είναι καλύτερα και πιο κατανοητά. Με λίγη εξάσκηση μπορείτε να φτάσετε σχετικά πολύ γρήγορα σε ένα λογικό“ακατέργαστο” περίγραμμα.
