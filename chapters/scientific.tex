Φυσικά, μια ανώτερη εξωτερική μορφή από μόνη της δεν είναι και μια καλή βαθμολογία.
Μάλλον μια καλή εργασία πρέπει να διέπεται από μια καλή επιστημονική μεθοδολογία. Στο κέντρο αυτής της μεθοδολογίας βρίσκονται τρεις όροι: Ανάλυση, Σχεδιασμός και η Εφαρμογή.
Ένα δεδομένο πρόβλημα πρέπει αρχικά να αναλυθεί προσεκτικά. Σε ποια επιμέρους προβλήματα διασπάται; Ποιος είναι ο πυρήνας του προβλήματος; Στη συνέχεια πρέπει  στα πλαίσια του σχεδιασμού να αναπτύξουμε προσεγγίσεις λύσεων. Αυτό γίνεται συχνά σε ένα πιο αφηρημένο επίπεδο, χωρίς αναφορά σε συγκεκριμένες τεχνολογίες ή προϊόντα. Τέλος, ακολουθεί η πραγματική εφαρμογή. Όλα αυτά γίνονται, φυσικά, ακόμη πριν από τη σύνταξη μιας εργασίας. Ωστόσο, αυτή διαδικασία πρέπει να αναδείξει τους σταθμούς μιας επιστημονικής προσέγγισης. Ανάλυση, σχεδίαση και υλοποίηση θα πρέπει να παρουσιαστούν διακριτά. Σε γενικές γραμμές, η έμφαση της εργασίας βρίσκετε στην ίδια τη φάση του σχεδιασμού. 
Η τελική εφαρμογή επιβάλλεται περισσότερο για να αποδείξουμε ότι το μοντέλο όπως σχεδιάστηκε, είναι βιώσιμο. Αυτό ισχύει ιδιαίτερα για τις πτυχιακές εργασίες. Σε συντομότερες εργασίες το έργου μετατοπίζεται περισσότερο στην εφαρμογή, ενώ ο σχεδιασμός συνεχίζει να διαδραματίζει σημαντικό ρόλο.
Υπάρχουν κάποιοι κανόνες των οποίων η τήρηση μπορεί να συμβάλει σημαντικά στην ποιότητα της σχεδίασης. Παρακάτω θα εξετάσουμε εν συντομία ορισμένους από αυτούς τους κανόνες.
Κατά τη διάρκεια της προκαταρκτικής εργασίας, λαμβάνονται αποφάσεις σε πολλά σημεία, οι οποίες επηρεάσουν τη τελική μορφή των μοντέλων, την εφαρμογή και, τέλος, επίσης τον συνολικό σχεδιασμό. Εσείς αποφασίζετε για ορισμένα μοντέλα και τεχνολογίες. Τεκμηριώστε στα κατάλληλα σημεία του εγγράφου σας αυτές τις αποφάσεις.
Σε μια συζήτηση που ακολουθεί συχνά ακούμε δηλώσεις όπως:
``Έχω αποφασίσει για XML, επειδή έτσι αναφερόταν στην εκπόνηση του έργου.''
Αυτό δείχνει πάντα ότι η δουλειά σας δεν αντικατοπτρίζεται επαρκώς στην εργασία. 
Βρείτε αντικειμενικούς λόγους για τη λήψη αποφάσεων. Εάν δεν υπάρχουν προφανείς λόγοι, αμφισβητήστε την εν λόγω απόφαση και αναζητήστε εναλλακτικές λύσεις.
Μην αφήνετε αναπάντητα ερωτήματα. Αν έχετε καλές απαντήσεις στα ερωτήματα που τίθενται μη διστάσετε να τις αναδείξετε. Δεν υπάρχουν καλές απαντήσεις, τότε να τονίσετε αυτό το γεγονός ανοιχτά. Επισημάνετε το γεγονός και δικαιολογήστε το. Σε γενικές γραμμές, διευκρινίζουμε τέτοια σημεία πολύ πριν από τη σύνταξη της εργασίας με τον επιβλέποντα. Τα ανοικτές ζητήματα αναδύονται πάντα. Σκεφτείτε αν θα πρέπει να είναι αντικείμενο της εργασίας σας.
Επίσης ανοιχτά θα πρέπει να ασχοληθείτε και με τις αδυναμίες της λύσης σας. Τίποτα  και κανείς δεν είναι τέλειος. Δείξτε τα αδύναμα σημεία και να προσπαθήστε να εντοπίσετε πιθανές λύσεις. Αυτό δείχνει ότι είστε σε θέση να εξετάζετε την εργασία σας με κριτικό πνεύμα και να μαθαίνετε από τα λάθη. Εάν η λύση σας έχει αδυναμίες, αυτό δεν σημαίνει ότι η βαθμολογία σας πρέπει να είναι κακή. Εάν παραβλέπετε τα αδύνατα σημεία ή τα αποκρύπτετε τότε αφήνετε περιθώρια για αρνητικά συμπεράσματα για τις μεθόδους της εργασίας σας.
Τα εξαίρετα  έργα χαρακτηρίζονται συχνά από το γεγονός ότι ο συγγραφέας έχει ρίξει ματιές πέρα του στενού θέματός του. Μια υπερβολικά μεγάλη συγκέντρωση στο στενό θέμα παρεμποδίζει συχνά την αντίληψη για το γενικό. Θα πρέπει να κατατάξετε το έργο σας σε ένα ευρύτερο πλαίσιο. Θα πρέπει να είστε σε θέση να αξιολογήσετε τομείς που σχετίζονται με το θέμα σας και δείχνουν τη σχέση τους με τη δική σας εργασία.
