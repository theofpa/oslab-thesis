Για την στοιχειοθεσία του κειμένου υπάρχει ένας αριθμός εναλλακτικών συστημάτων. Εφαρμογές του Office όπως το OpenOffice ή LibreOffice αφήνουν στην ευχέρεια του δημιουργού πολλή δουλειά για τη μορφοποίηση του κειμένου. Ειδικότερα, η ορθή ενσωμάτωση της βιβλιογραφίας και παραπομπών, καθώς και η τοποθέτηση και συσχέτιση των εικόνων και πινάκων αποδεικνύεται συχνά δύσκολη. Εάν χρησιμοποιήσετε αυτά τα προϊόντα, μπορείτε να επιλέξετε την προεπιλεγμένη ρύθμιση για τα νέα έγγραφα και να την τροποποιήσετε κατάλληλα, ενδεχομένως, ώστε να ανταποκρίνονται στο στυλ οδηγιών μας.
Μέχρι στιγμής έχουμε πολύ καλή εμπειρία με το σύστημα \LaTeX{} κατά την συγγραφή μεγάλων και σύνθετων κειμένων όπως πτυχιακές.
Το σύστημα αυτό βασίζεται στη δική του σύνταξη μορφοποίησης του κειμένου. Τα έγγραφα δημιουργούνται σε μορφή απλού ASCII κειμένου με ένα απλό πρόγραμμα επεξεργασίας και στη συνέχεια να μετατρέπεται σε ένα αρχείο PostScript ή PDF.
Εδώ, ο μεταφραστής επιλέγει αυτόματα όλες τις παραμέτρους στοιχειοθεσίας κειμένου σωστά. Ιδιαίτερα αξιοσημείωτο είναι ότι το \LaTeX{} (ή το βασικό σύστημα \TeX{}) διαθέτει ένα πολύ ισχυρό και εύκολο στη χρήση  σύστημα παραπομπών και αναφοράς.
Υπάρχουν διάφορες κατηγορίες εγγράφων στη διάθεσή σας, δίνοντας στα έγγραφά σας μια προκαθορισμένη εμφάνιση, η οποία είναι επίσης από τυπογραφικής πλευράς άψογη. Οι κατηγορίες αυτές αντικαθίστανται εύκολα, χωρίς την ανάγκη αλλαγής του συνολικού εγγράφου. Τέλος, το σύστημα στοιχειοθεσίας μαθηματικών τύπων του LaTeX είναι σε πρωτόγνωρα επίπεδα. Για τους λόγους αυτούς, η πλειονότητα των επιστημονικών δημοσιεύσεων γίνεται με \LaTeX{} ή παρόμοια συστήματα.
Το \LaTeX{} διατίθεται ελεύθερα για όλες τις πλατφόρμες και είναι μέρος της προεπιλεγμένης εγκατάστασης στα συστήματα Linux. Για περισσότερες πληροφορίες, ανατρέξτε για παράδειγμα στην ιστοσελίδα \url{http://www.latex-project.org/}.
