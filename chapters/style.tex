\section{Γενικά}
Μια εργασία είναι πάντα ένα ολοκληρωμένο και συσχετισμένο κείμενο. Η αριθμημένη λίστα ή στυλ κουκκίδας πρέπει να χρησιμοποιείται με φειδώ. Να γράφετε ολόκληρη την πρόταση και να βεβαιωνόσαστε ότι δεν είναι πολύ μεγάλη. Αποφύγετε περίπλοκες προτάσεις. Επίσης, αποφύγετε τους αργκό όρους. Η χρήση του α' ενικού  γενικά θεωρείται κακή τεχνική και θα πρέπει να αποφεύγεται όποτε είναι δυνατόν. Θεωρείται επίσης κακή συνήθεια η υπέρμετρη χρήση  των παθητικών γλωσσικών κατασκευών.
\section{Αλλοδαποί φοιτητές}
Οι αλλοδαποί φοιτητές υπόκεινται στις ίδιες απαιτήσεις ως προς τη μορφή, τη δομή και τη γλώσσα με τους Έλληνες σπουδαστές . Ωστόσο, το γεγονός ότι η γλώσσα δεν είναι η μητρική τους γλώσσα κατά την εκτίμηση θα παίρνεται καλοπροαίρετα υπόψη.
\section{Η ομάδα-στόχος}
Κάθε γραπτή εργασία έχει ένα συγκεκριμένο ακροατήριο. Να ξέρετε σε ποιόν απευθύνεστε. Είναι πολύ σημαντικό το κείμενο να είναι ειδικά προσαρμοσμένο στη σχετική κατηγορία αναγνωστών. Γράφετε μια εύκολα κατανοητή περίληψη για τα ανώτερα στελέχη (``Executive Summary''), γράφετε για τους εμπειρογνώμονες, ή γράφετε για τους επαγγελματίες οι οποίοι προέρχονται από συναφή θέματα;
Σε γενικές γραμμές, θα πρέπει οι εργασίες που αναπτύσσονται στα πλαίσια των μαθημάτων να είναι κατανοητές για τους αναγνώστες που είναι επαγγελματίες αλλά μπορεί να μη έχουν την απαιτούμενη βαθύτερη κατανόηση των ειδικών εδαφίων της εργασίας. Κατά συνέπεια, οι εργασίες θα είναι τεχνικά προσανατολισμένες. Ωστόσο, στα εισαγωγικά τους κεφάλαια  θα θίγονται όλα τα ειδικά θέματα που έχουν σημασία για την κατανόηση τους. Παραπέρα θα γίνονται αναφορές σε εργασίες που η μελέτη τους δίνει μια πιο βαθιά ματιά στις περιοχές αυτές.
