\section{Σελίδα Τίτλου}
Η σελίδα τίτλου ενός έργου πρέπει να περιέχει όλα τα βασικά στοιχεία σχετικά με το ίδιο το έργο αλλά και τον συγγραφέα. Αυτά έχουν ως εξής:
\begin{itemize}
\item Ο τίτλος του έργου
\item Ο τίτλος του μαθήματος
\item Η φύση του μαθήματος (έργο, σεμινάρια, κλπ.)
\item Το εξάμηνο (για παράδειγμα, εξάμηνο καλοκαίρι 2010)
\item Το όνομα του επιβλέποντος
\item Ονοματεπώνυμο και αριθμός μητρώου του συγγραφέα (με ομαδική εργασία ισχύει για όλα τα μέλη της ομάδας)
\item Η ημερομηνία
\end{itemize}
Σε περίπτωση πτυχιακών εργασιών πρέπει να εμπεριέχονται τα ακόλουθα στοιχεία:
\begin{itemize}
\item Ο τίτλος  “Πτυχιακή Εργασία”
\item Ο τίτλος του έργου
\item "Παρουσιάστηκε από"
\item Το όνομα του σπουδαστή
\item Τόπος και ημερομηνία
\item "Ο Επιβλέπων:"
\item Όλο το ονοματεπώνυμο του επιβλέποντα
\item Εργαστήριο
\item Τμήμα
\item Σχολή
\item Ίδρυμα
\end{itemize}

\section{Περίγραμμα Περιεχομένου}
Η διάκριση τμημάτων κειμένου έχει αποφασιστική σημασία. Η δομή ενός έργου αποτελεί τον καθοριστικό παράγοντα για το πόσο καλά μπορεί κανείς να ακολουθήσει το κείμενο και κατά πόσο γίνεται το περιεχόμενο κατανοητό. Μια εργασία θα πρέπει να αποτελείται συνήθως από τα ακόλουθα τμήματα.
\subsection{Εισαγωγή}
Η εισαγωγή έχει ως στόχο να εξοικειώσει τον αναγνώστη γύρο από το θέμα της εργασίας. Στόχος είναι να πάρει μια γενική εικόνα του περιεχομένου. Στην εισαγωγή, παρουσιάζεται το υπό αντιμετώπιση πρόβλημα και εξηγείται σε ένα υψηλού επιπέδου αφαίρεση. Στο τέλος της εισαγωγής, πρέπει να δίνεται μια περίληψη για το περαιτέρω περιεχόμενο. Εδώ, κατονομάζονται σύντομα τα επιμέρους τμήματα της εργασίας και  τοποθετούνται στο γενικό τους πλαίσιο. Έτσι παρέχεται στον αναγνώστη το "νήμα" της εργασίας.
\subsection{Περιγραφή προβλήματος/Κίνητρο}
Σε μια μεγαλύτερης έκτασης εργασία μπορεί να είναι χρήσιμο να εισαχθεί ένα ειδικό κεφάλαιο σχετικά με το πρόβλημα. Εδώ, θα πρέπει πάλι χωριστά και με ακρίβεια να παρουσιάσετε το υπό μελέτη πρόβλημα.
\subsection{Θεμελίωση}
Όλες οι επιστημονικές εργασίες βασίζονται σε άλλες εργασίες. Υπάρχουν ορισμένα θεμελιώδη στοιχεία που είναι απαραίτητα για την κατανόηση του θέματος. Με το τμήμα αυτό ο αναγνώστης θα πρέπει να είναι σε θέση να μπορεί επιστημονικά να ακολουθεί την υπόλοιπη εργασία. Ανάλογα με την έκταση των υπό παρουσίαση θεμελιωδών εννοιών το τμήμα αυτό μπορεί να χωριστεί σε περαιτέρω υποενότητες.
\subsection{Σχετικές εργασίες}
Μια ακαδημαϊκή μελέτη εντάσσεται πάντα σε ένα μεγαλύτερο επιστημονικό πλαίσιο. Δεν στέκεται ποτέ μόνη της. Υπάρχουν και άλλες εργασίες, που ασχολούνται με το ίδιο πρόβλημα αλλά το προσεγγίζουν με άλλο τρόπο. Ορισμένες μπορεί και επιλύουν το πρόβλημα με παρόμοιο τρόπο. Για λόγους που σχετίζονται με την κατανόηση αλλά και για λόγους επιστημονικής κατάταξης είναι απαραίτητη η ενασχόληση με τις σχετικές εργασίες. Αυτό ισχύει ιδιαίτερα για τις εργασίες πτυχιακών σπουδών.
\subsection{Κύριο μέρος}
Αφού ο αναγνώστης γνωρίζει τώρα το κύριο νήμα, έχει εισαχθεί στο θέμα και οι βασικές αρχές είναι γνωστές, ακολουθεί η αντιμετώπιση του πραγματικού προβλήματος. Το σώμα μιας εργασίας ανάλογα με το αντικείμενο πρέπει να χωρίζεται σε κατάλληλα επιμέρους τμήματα. Η ακριβής του δομή γενικά δύσκολα μπορεί να εντοπιστεί διότι εξαρτάται στενά από το ίδιο το περιεχόμενο. Ωστόσο, διέπεται συνήθως από τη γενική επιστημονική προσέγγιση. Δηλαδή, κατ' αρχήν πρέπει να γίνει μια λεπτομερή ανάλυση του προβλήματος. Από αυτή την ανάλυση πηγάζει και ο σχεδιασμός της λύσης. Τέλος, ακολουθεί η εφαρμογή αυτού του σχεδιασμού σε μια συγκεκριμένη υλοποίηση. Τα βήματα της Ανάλυσης - Σχεδιασμού - Υλοποίησης είναι μια καλή κατευθυντήρια γραμμή για την ανάπτυξη του κύριου μέρους της εργασίας.
\subsection{Ανακεφαλαίωση}
Ανάλογα με τον όγκο της εργασίας, θα πρέπει στο τέλος του κύριου μέρους να ακολουθεί μια σύντομη ανακεφαλαίωση του περιεχομένου. Είναι πολύ σημαντικό μέρος του τελικού τμήματος της εργασίας. Σε μικρότερης έκτασης έργα, η ανακεφαλαίωση αυτή μπορεί να συγχωνευθεί με το τμήμα για την αξιολόγηση-εκτίμηση.
\subsection{Εκτίμηση}
Η εκτίμηση των ίδιων αποτελεσμάτων αλλά και τα αποτελέσματα άλλων πρέπει να εξεταστούν και να αξιολογηθούν κριτικά που αποτελεί την ουσία  μιας επιστημονικής εργασίας. Κάθε σύστημα και κάθε λύση έχει πλεονεκτήματα και αδυναμίες. Ποια είναι αυτά; Σε ποιο βαθμό οι στόχοι έχουν επιτευχθεί; Αυτά τα ερωτήματα πρέπει να απαντηθούν.
\subsection{Ανοιχτά ζητήματα}
Δεν το έργο μπορεί να συζητήσει ένα θέμα εξαντλητικά. Υπάρχουν πάντα ερωτήματα και  σημεία για περαιτέρω εξέταση. Ίσως οι αρχικές προβλέψεις δεν έχουν εκπληρώσει αυτά που είχαν υποσχεθεί και κατά την εξέλιξη της εργασίας παρουσιάστηκαν εναλλακτικές λύσεις. Σ' αυτό το κεφάλαιο αυτά τα ζητήματα πρέπει οπωσδήποτε να παρουσιαστούν.
\subsection{Βιβλιογραφία/Αναφορές}
Όλες οι επιστημονικές εργασίες κάνουν αναφορές σε ήδη υπάρχοντα έργα.
Αυτές θα πρέπει να σημειώνονται στο κείμενο στην κατάλληλη θέση όπως και στο τέλος της εργασίας  όπου συλλέγονται σε μορφή καταλόγου βιβλιογραφίας/αναφορών άρθρων περιοδικών, συνεδρίων καθώς και διαδικτυακών αναφορών.
\subsection{Περαιτέρω τμήματα}
Μεγάλες εργασίες (πχ. διπλωματικές και διδακτορικές διατριβές) περιέχουν περισσότερους  διαφορετικούς καταλόγους. Ο πίνακας περιεχομένων, Πίνακας Σχημάτων ακόμα και των ίδιων των Πινάκων όπως και Πίνακες Συντμήσεων συνήθως παραθέτονται στην αρχή της εργασίας. Το Γλωσσάριο και ευρετήριο (αν είναι απαραίτητο ή επιθυμητό) το προσθέτουμε στο τέλος του εγγράφου.
\subsection{Παράρτημα}
Ογκώδης τμήματα του έργου, π.χ. εκτενής πηγαίος κώδικας λογισμικού ή μεγάλα διαγράμματα και σχέδια τοποθετούνται στα σχετικά παραρτήματα και αναφέρονται στο κείμενο της εργασίας ανάλογα.
Δεν είναι όλα τα πιο πάνω στοιχεία απαραίτητα να συμπεριλαμβάνονται σε κάθε εργασία. Η σειρά μπορεί να ποικίλλει υπό ορισμένες προϋποθέσεις. Πίνακας 1 βασίζεται σε τρεις τύπους μακροπρόθεσμης προπαρασκευής χαρτί, έκθεση για το έργο / δοκίμιο και η διατριβή καθορισθεί σαφώς ποια μέρη του έργου πρέπει να περιέχει. Εδώ χρησιμοποιούμε διαβαθμίσεις σπουδαιότητας είναι χρήσιμο να έχει σημασία. Ορισμένα πεδία περιέχουν δηλώσεις υπό όρους, ανάλογα με το πραγματικό περιεχόμενο της εργασίας. 
Οι τίτλοι των επιμέρους τμημάτων θα πρέπει να είναι όσο πιο κατανοητοί γίνεται. Έτσι, για παράδειγμα, πρέπει να αποφεύγεται, ότι το κύριο σώμα της εργασίας απλά να αποκαλείται ως "κύριο μέρος».

\section{Έκταση}
\label{sec:έκταση}
Μια κατάλληλη γραπτή εργασία έχει ένα ορισμένο όγκο. Δεν είναι δυνατόν να φιλοξενηθεί μια καλή έκθεση έργου σε μόνο 5 σελίδες ενώ σε μια πτυχιακή εργασία η αντιμετώπιση του γνωστικού αντικειμένου της εκτείνεται συνήθως σε περισσότερο από 50 σελίδες για να μπορεί να ολοκληρωθεί. Ο Πίνακας 2 περιέχει πληροφορίες σχετικά με το κατάλληλο όγκο του έργου (σε σελίδες). Τα επιμέρους στοιχεία αναφέρονται εδώ ως ποσοστό επί του συνόλου του έργου. Προσοχή, όλα τα κεφάλαια πριν από το “κύριο μέρος” τα εντάσσουμε στο "εισαγωγικό τμήμα" ενώ όλα τα υπόλοιπα κεφάλαια μετά το “κύριο μέρος” τα χαρακτηρίζουμε ως το "τελευταίο τμήμα".
Ειδικά σε σχέση με τις πτυχιακές εργασίες ο Πίνακας 2 δίνει μόνο τις κατά προσέγγιση τιμές. Αν με τους δικούς σας υπολογισμούς και στόχους οι εν λόγω τιμές διαφέρουν έντονα, θα πρέπει να προηγηθεί μια κατά βάθος συζήτηση με τον συγκεκριμένο επόπτη σας κατά τη διάρκεια της εκπόνησης της εργασίας, και να καταλήξει σε μια συμφωνία αποδεκτή και απ' τους δύο.
Εργασίες προπτυχιακών μαθημάτων έχουν γενικά ένα πιο ισχυρό πρακτικό προσανατολισμό σε σχέση με τις πτυχιακές ή ακόμα και μεταπτυχιακές εργασίες. Έτσι, έρχεται εδώ η πρακτική λύση ενός συγκεκριμένου προβλήματος και η τεχνική εφαρμογή σε πρώτο πλάνο, ενώ η επιστημονική ένταξη και η μεθοδολογική ανάπτυξη θεμελιωδών σκεπτικών και συλλήψεων έχουν κάπως μικρότερη βαρύτητα, ιδιαίτερα όσο αφορά την έκταση των αντίστοιχων τμημάτων. Δεν χρειάζεται να αναφερθεί, ότι και σ' αυτή τη περίπτωση πρόκειται εξίσου για μια επιστημονική εργασία, έτσι ώστε να παραχωρηθεί στα πιο άνω τμήματα μια σχετική σημασία και να ενταχθούν κι αυτά κατάλληλα στο συνολικό έργο.


\begin{table}[ht]
\centering
%\small\addtolength{\tabcolsep}{-3pt}
\begin{tabular}{ | p{2.8cm} | p{3cm} | p{3cm} | p{3cm} |}
\hline
 & Ασκήσεις Θεωρίας / Εργαστηρίου & Αναφορές έργων / εργασίες εξαμήνου & Πτυχιακές / Μεταπτυχιακές εργασίες \\ \hline
 Κατάλογος Περιεχομένων & ασήμαντο & \textcolor{green}{χρήσιμο} & \textcolor{red}{σημαντικό} \\ \hline
 Κατάλογος Ακρωνύμων & ασήμαντο & ασήμαντο & \textcolor{green}{χρήσιμο} \\ \hline
 Ευρετήριο εικόνων & ασήμαντο & ασήμαντο & \textcolor{green}{χρήσιμο} \\ \hline
 Ευρετήριο πινάκων & ασήμαντο & ασήμαντο & \textcolor{green}{χρήσιμο} \\ \hline
 Εισαγωγή & \textcolor{red}{σημαντικό} & \textcolor{red}{σημαντικό} & \textcolor{red}{σημαντικό} \\ \hline
 Περιγραφή προβλήματος και κίνητρα & Ενδεχομένη  ενσωμάτωση  στην εισαγωγή & Ενδεχομένη  ενσωμάτωση  στην εισαγωγή & \textcolor{red}{σημαντικό} \\ \hline
 Θεμελίωση & \textcolor{red}{σημαντικό} & \textcolor{red}{σημαντικό} & \textcolor{red}{σημαντικό} \\ \hline
 Σχετικές εργασίες & \textcolor{red}{σημαντικό} & \textcolor{green}{χρήσιμο} & \textcolor{red}{σημαντικό} \\ \hline
 Κύριο μέρος & \textcolor{red}{σημαντικό} & \textcolor{red}{σημαντικό} & \textcolor{red}{σημαντικό} \\ \hline
 Περίληψη / ανακεφαλαίωση & \textcolor{red}{σημαντικό} & \textcolor{red}{σημαντικό} & \textcolor{red}{σημαντικό} \\ \hline
 Εκτίμηση αποτελεσμάτων & \textcolor{red}{σημαντικό} & \textcolor{red}{σημαντικό} & \textcolor{red}{σημαντικό} \\ \hline
 Ανοιχτά ζητήματα & \textcolor{green}{χρήσιμο} & \textcolor{red}{σημαντικό} & \textcolor{red}{σημαντικό} \\ \hline
 Γλωσσάριο & ασήμαντο & \textcolor{green}{χρήσιμο} & \textcolor{green}{χρήσιμο} \\ \hline
 Ευρετήριο & ασήμαντο & ασήμαντο & \textcolor{green}{χρήσιμο} \\ \hline
 Βιβλιογραφία / Διαδικτυακές Αναφορές & \textcolor{red}{σημαντικό} & \textcolor{red}{σημαντικό} & \textcolor{red}{σημαντικό} \\ \hline
 Παραρτήματα & Δεν απαιτείται & Κατά περίπτωση & Κατά περίπτωση \\ \hline
\end{tabular}
\caption{Σπουδαιότητα των επιμέρους στοιχείων ανάλογα με τον τύπο της εργασίας}
\label{tab:importance}
\end{table}

\begin{table}[ht]
\centering
%\small\addtolength{\tabcolsep}{-3pt}
\begin{tabular}{ | p{2.5cm} | p{2.3cm} | p{2.3cm} | p{2.3cm} | p{3.2cm} | }
\hline
 & Ασκήσεις Θεωρίας / Εργαστηρίου & Αναφορές  έργων / εργασίες εξαμήνου & Πτυχιακές εργασίες & Μεταπτυχιακές εργασίες \\ \hline
Εισαγωγικό τμήμα & 20.00\% & 15\% - 20\% & 15\% - 20\% & 15\% - 20\% \\ \hline
Κύριο τμήμα & 70.00\% & 15\% - 20\% & 15\% - 20\% & 15\% - 20\% \\ \hline
Τελευταίο τμήμα & 10.00\% & 10.00\% & 10.00\% & 10.00\% \\ \hline
Συνολική έκταση (σελίδες) & 10/15/11 & 15 - 25 & 50 - 80 & 80 - 130 \\ \hline
\end{tabular}
\caption{Τυπικός όγκος εργασιών σπουδαστών και ποσοστιαία κατανομή}
\label{tab:importance}
\end{table}
