Πολλά μαθήματα απαιτούν για την ολοκλήρωσή τους την εκτέλεση μιας γραπτής εργασίας. Για παράδειγμα στα πλαίσια ενός σεμιναριακού μαθήματος ο σπουδαστής καλείται να επεξεργαστεί με βάση άλλες πρότυπες εργασίες ένα επιστημονικό θέμα και να το παρουσιάσει.
Στις τελικές εκθέσεις έργων επιβάλλεται οι επεξεργασμένες ιδέες να συζητιούνται, καθώς και η ανάπτυξη σχετικού λογισμικού να είναι τεκμηριωμένη. Όσο διαφορετικές και να είναι οι απαιτήσεις ως προς το περιεχόμενο, η εργασία επιβάλλεται να υπακούει σε γενικούς κανόνες μιας εξωτερικής εμφάνισης. Αν μια εργασία λόγω της κακής δόμησης, υπερβολικού αριθμού ορθογραφικών και γραμματικών λαθών και αδύναμων διατυπώσεων, διαβάζεται με δυσκολία, τότε είναι αδύνατο για τον βαθμολογητή να κάνει σωστή εκτίμηση των τεχνικών και επιστημονικών επιτευγμάτων του σπουδαστή. Όποιος δεν είναι σε θέση να παρουσιάσει σε κατάλληλη γραπτή μορφή τις επιδόσεις του στην ανάπτυξη σύνθετων εννοιών, θα παίρνει συχνά κακούς βαθμούς. Επιπλέον θα περίμενε κανείς από έναν απόφοιτο ΑΕΙ κάποια ευχέρεια λόγου, μια ικανότητα που θα πρέπει να έχει ήδη αποκτήσει πριν την εισαγωγή του στο τριτοβάθμιο ίδρυμα είτε είναι αυτό το πανεπιστήμιο είτε ΤΕΙ. Όλα αυτά είναι ποιότητες και χαρακτηριστικά, οι οποίες είναι επίσης χρήσιμες για την επαγγελματική ζωή ενός αποφοίτου. Πρέπει να είναι σε θέση να συντάσσει ακόμα πιο μεγάλα έγγραφα γρήγορα και ευκατανόητα.
Στα στοιχεία αποτίμησης της κάθε εργασίας ανήκουν ως εκ τούτου το περιεχόμενο, η χρησιμοποιούμενη γλώσσα και η εξωτερική μορφή. Σε αυτόν τον οδηγό περιγράφονται εν συντομία οι βασικές απαιτήσεις για την δομή και την εξωτερική εμφάνιση. Επιπλέον, παραθέτουμε μια σύντομη εισαγωγή για την ανάπτυξη μιας διατριβής καθώς και για την επιστημονική έρευνα γενικότερα.
Στην πράξη συνήθως η συγγραφή ενός έργου άθελα υποτιμάται. Όπως και στην περίπτωση ανάπτυξης ενός τεχνικού συστήματος, ο σχεδιασμός ενός επιστημονικού κειμένου, απαιτεί μια μεθοδολογική προσέγγιση, μια συνήθως χρονοβόρα διαδικασία ανάπτυξης. Το παρόν έγγραφο βασίζεται στις σχετικές προδιαγραφές που αναπτύχθηκαν στα αντίστοιχα πολυτεχνικά ιδρύματα της Γερμανίας όπως TU Βερολίνου, Πανεπιστήμιο της Στουτγάρδης και TU Chemnitz.
Για το Εργαστήριο Λειτουργικών Συστημάτων και τα μαθήματα που διεξάγονται απ' αυτό αποτελεί το πρότυπο παραδοτέας εργασίας οποιουδήποτε μεγέθους και είναι δεσμευτικό για τους σπουδαστές και το εκπαιδευτικό προσωπικό του.
